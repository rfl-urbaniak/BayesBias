% Options for packages loaded elsewhere
\PassOptionsToPackage{unicode}{hyperref}
\PassOptionsToPackage{hyphens}{url}
\PassOptionsToPackage{dvipsnames,svgnames,x11names}{xcolor}
%
\documentclass[
  10pt,
  dvipsnames,enabledeprecatedfontcommands, twocolumn]{scrartcl}
\title{Taking uncertainty in word embedding bias estimation seriously: a
bayesian approach}
\author{Alicja Dobrzeniecka and Rafal Urbaniak}
\date{}

\usepackage{amsmath,amssymb}
\usepackage{lmodern}
\usepackage{iftex}
\ifPDFTeX
  \usepackage[T1]{fontenc}
  \usepackage[utf8]{inputenc}
  \usepackage{textcomp} % provide euro and other symbols
\else % if luatex or xetex
  \usepackage{unicode-math}
  \defaultfontfeatures{Scale=MatchLowercase}
  \defaultfontfeatures[\rmfamily]{Ligatures=TeX,Scale=1}
\fi
% Use upquote if available, for straight quotes in verbatim environments
\IfFileExists{upquote.sty}{\usepackage{upquote}}{}
\IfFileExists{microtype.sty}{% use microtype if available
  \usepackage[]{microtype}
  \UseMicrotypeSet[protrusion]{basicmath} % disable protrusion for tt fonts
}{}
\makeatletter
\@ifundefined{KOMAClassName}{% if non-KOMA class
  \IfFileExists{parskip.sty}{%
    \usepackage{parskip}
  }{% else
    \setlength{\parindent}{0pt}
    \setlength{\parskip}{6pt plus 2pt minus 1pt}}
}{% if KOMA class
  \KOMAoptions{parskip=half}}
\makeatother
\usepackage{xcolor}
\IfFileExists{xurl.sty}{\usepackage{xurl}}{} % add URL line breaks if available
\IfFileExists{bookmark.sty}{\usepackage{bookmark}}{\usepackage{hyperref}}
\hypersetup{
  pdftitle={Taking uncertainty in word embedding bias estimation seriously: a bayesian approach},
  pdfauthor={Alicja Dobrzeniecka and Rafal Urbaniak},
  colorlinks=true,
  linkcolor={Maroon},
  filecolor={Maroon},
  citecolor={Blue},
  urlcolor={blue},
  pdfcreator={LaTeX via pandoc}}
\urlstyle{same} % disable monospaced font for URLs
\usepackage{longtable,booktabs,array}
\usepackage{calc} % for calculating minipage widths
% Correct order of tables after \paragraph or \subparagraph
\usepackage{etoolbox}
\makeatletter
\patchcmd\longtable{\par}{\if@noskipsec\mbox{}\fi\par}{}{}
\makeatother
% Allow footnotes in longtable head/foot
\IfFileExists{footnotehyper.sty}{\usepackage{footnotehyper}}{\usepackage{footnote}}
\makesavenoteenv{longtable}
\usepackage{graphicx}
\makeatletter
\def\maxwidth{\ifdim\Gin@nat@width>\linewidth\linewidth\else\Gin@nat@width\fi}
\def\maxheight{\ifdim\Gin@nat@height>\textheight\textheight\else\Gin@nat@height\fi}
\makeatother
% Scale images if necessary, so that they will not overflow the page
% margins by default, and it is still possible to overwrite the defaults
% using explicit options in \includegraphics[width, height, ...]{}
\setkeys{Gin}{width=\maxwidth,height=\maxheight,keepaspectratio}
% Set default figure placement to htbp
\makeatletter
\def\fps@figure{htbp}
\makeatother
\setlength{\emergencystretch}{3em} % prevent overfull lines
\providecommand{\tightlist}{%
  \setlength{\itemsep}{0pt}\setlength{\parskip}{0pt}}
\setcounter{secnumdepth}{5}
\newlength{\cslhangindent}
\setlength{\cslhangindent}{1.5em}
\newlength{\csllabelwidth}
\setlength{\csllabelwidth}{3em}
\newlength{\cslentryspacingunit} % times entry-spacing
\setlength{\cslentryspacingunit}{\parskip}
\newenvironment{CSLReferences}[2] % #1 hanging-ident, #2 entry spacing
 {% don't indent paragraphs
  \setlength{\parindent}{0pt}
  % turn on hanging indent if param 1 is 1
  \ifodd #1
  \let\oldpar\par
  \def\par{\hangindent=\cslhangindent\oldpar}
  \fi
  % set entry spacing
  \setlength{\parskip}{#2\cslentryspacingunit}
 }%
 {}
\usepackage{calc}
\newcommand{\CSLBlock}[1]{#1\hfill\break}
\newcommand{\CSLLeftMargin}[1]{\parbox[t]{\csllabelwidth}{#1}}
\newcommand{\CSLRightInline}[1]{\parbox[t]{\linewidth - \csllabelwidth}{#1}\break}
\newcommand{\CSLIndent}[1]{\hspace{\cslhangindent}#1}
%\documentclass{article}

% %packages
\usepackage{booktabs}
%\usepackage[left]{showlabels}
\usepackage{multirow}
\usepackage{subcaption}

\usepackage{graphicx}
\usepackage{longtable}
\usepackage{ragged2e}
\usepackage{etex}
%\usepackage{yfonts}
\usepackage{marvosym}
\usepackage[notextcomp]{kpfonts}
\usepackage{nicefrac}
\newcommand*{\QED}{\hfill \footnotesize {\sc Q.e.d.}}
\usepackage{floatrow}

\usepackage[textsize=footnotesize]{todonotes}
\newcommand{\ali}[1]{\todo[color=gray!40]{\textbf{Alicja:} #1}}
\newcommand{\mar}[1]{\todo[color=blue!40]{#1}}
\newcommand{\raf}[1]{\todo[color=olive!40]{#1}}

%\linespread{1.5}
\newcommand{\indep}{\!\perp \!\!\! \perp\!}


\setlength{\parindent}{10pt}
\setlength{\parskip}{1pt}


%language
%\usepackage{times}
\usepackage{mathptmx}
\usepackage[scaled=0.86]{helvet}
\usepackage{t1enc}
%\usepackage[utf8x]{inputenc}
%\usepackage[polish]{babel}
%\usepackage{polski}




%AMS
\usepackage{amsfonts}
\usepackage{amssymb}
\usepackage{amsthm}
\usepackage{amsmath}
\usepackage{mathtools}

\usepackage{geometry}
 \geometry{a4paper,left=35mm,top=20mm,}


%environments
\newtheorem{fact}{Fact}



%abbreviations
\newcommand{\ra}{\rangle}
\newcommand{\la}{\langle}
\newcommand{\n}{\neg}
\newcommand{\et}{\wedge}
\newcommand{\jt}{\rightarrow}
\newcommand{\ko}[1]{\forall  #1\,}
\newcommand{\ro}{\leftrightarrow}
\newcommand{\exi}[1]{\exists\, {_{#1}}}
\newcommand{\pr}[1]{\ensuremath{\mathsf{P}(#1)}}
\newcommand{\cost}{\mathsf{cost}}
\newcommand{\benefit}{\mathsf{benefit}}
\newcommand{\ut}{\mathsf{ut}}

\newcommand{\odds}{\mathsf{Odds}}
\newcommand{\ind}{\mathsf{Ind}}
\newcommand{\nf}[2]{\nicefrac{#1\,}{#2}}
\newcommand{\R}[1]{\texttt{#1}}
\newcommand{\prr}[1]{\mbox{$\mathtt{P}_{prior}(#1)$}}
\newcommand{\prp}[1]{\mbox{$\mathtt{P}_{posterior}(#1)$}}



\newtheorem{q}{\color{blue}Question}
\newtheorem{lemma}{Lemma}
\newtheorem{theorem}{Theorem}
\newtheorem{corollary}{Corollary}[fact]


%technical intermezzo
%---------------------

\newcommand{\intermezzoa}{
	\begin{minipage}[c]{13cm}
	\begin{center}\rule{10cm}{0.4pt}



	\tiny{\sc Optional Content Starts}
	
	\vspace{-1mm}
	
	\rule{10cm}{0.4pt}\end{center}
	\end{minipage}\nopagebreak 
	}


\newcommand{\intermezzob}{\nopagebreak 
	\begin{minipage}[c]{13cm}
	\begin{center}\rule{10cm}{0.4pt}

	\tiny{\sc Optional Content Ends}
	
	\vspace{-1mm}
	
	\rule{10cm}{0.4pt}\end{center}
	\end{minipage}
	}
	
	
%--------------------






















\newtheorem*{reply*}{Reply}
\usepackage{enumitem}
\newcommand{\question}[1]{\begin{enumerate}[resume,leftmargin=0cm,labelsep=0cm,align=left]
\item #1
\end{enumerate}}

\usepackage{float}

% \setbeamertemplate{blocks}[rounded][shadow=true]
% \setbeamertemplate{itemize items}[ball]
% \AtBeginPart{}
% \AtBeginSection{}
% \AtBeginSubsection{}
% \AtBeginSubsubsection{}
% \setlength{\emergencystretch}{0em}
% \setlength{\parskip}{0pt}






\usepackage[authoryear]{natbib}

%\bibliographystyle{apalike}



\usepackage{tikz}
\usetikzlibrary{positioning,shapes,arrows}

\ifLuaTeX
  \usepackage{selnolig}  % disable illegal ligatures
\fi

\begin{document}
\maketitle

\hypertarget{cosine-based-measures-of-bias}{%
\section{Cosine-based measures of
bias}\label{cosine-based-measures-of-bias}}

\raf{fix bibliography}

\ali{proofread and think if we should add more}

Modern Natural Language Processing (NLP) models are used to complete
various tasks such as providing email filters, smart assistants, search
results, language translations, text analytics and so on. All of them
need as an input words represented by means of numbers which is
accomplished with word embeddings. It seems that in the learning process
these models can learn implicit biases that reflect harmful
stereotypical thinking. One of the sources of bias in NLP can be located
in the way the word embeddings are made. There is a considerable amount
of literature available on the topic of bias detection and mitigation in
NLP models.

One of the first measures in the discussion has been developed by
Bolukbasi, Chang, Zou, Saligrama, \& Kalai (2016). There, the gender
bias of a word \(w\) is understood as its projection on the gender
direction \(\vec{w} \cdot (\overrightarrow{he} - \overrightarrow{she})\)
(the gender direction is the top principal compontent of ten gender pair
difference vectors). The underlying idea is that no bias is present if
non-explicitly gendered words are in equal distance to both elements in
all explicitly gender pairs. Given the (ideally) gender neutral words
\(N\) and the gender direction \(g\) the direct gender bias is defined
as the average distance of the words in \(N\) from \(g\) (\(c\) is a
parameter determining how strict we want to be): \begin{align}
\mathsf{directBias_c(N,g)} & = \frac{\sum_{w\in N}\vert \mathsf{cos}(\vec{w},g)\vert^c}{\vert N \vert }
\end{align}

The use of projections has been ciriticized for instance by Gonen \&
Goldberg (2019), who point out that while gender-direction might be an
indicator of bias, it is only one possible manifestation of it, and
reducing a projection of words might be insufficient. For instance,
``math'' and ``delicate'' might be in equal distance to both explicitly
gendered words while being closer to quite different stereotypical
attribute words. Further, the authors point out that most word pairs
preserve similarity under debiasing meant to minimize projection-based
bias.\footnote{Bolukbasi et al. (2016) use also another method which
  involves analogies and their evaluations by human users on Mechanical
  Turk. It is discussed and criticized in (Nissim, Noord, \& Goot,
  2020).}

To measure bias in word embeddings, Caliskan, Bryson, \& Narayanan
(2017) proposed the Word Embedding Association Test (WEAT). The idea is
that the measure of biases between two sets of target words, \(X\) and
\(Y\), (we call them protected words) should be quantified in terms of
the cosine similarity between the protected words and attribute words
coming from two sets of stereotype attribute words, \(A\) and \(B\)
(we'll call them attributes). For instance, \(X\) might be a set of male
names, \(Y\) a set of female names, \(A\) might contain stereotypically
male-related career words, and \(B\) stereotypically female-related
family words. WEAT is a modification of the Implicit Association Test
(IAT) (Nosek, Banaji, \& Greenwald, 2002) used in psychology and uses
almost the same word sets, allowing for a \emph{prima facie} sensible
comparison with bias in humans. If the person's attitude towards given
pair of concept is to be interpreted as neutral, there should be no
noticeable task completion time difference, and the final value from the
formula should be around 0. Let \(f\) be a similarity measure (usually,
cosine similarity). The association difference for a term \(t\) is:
\begin{align}
s(t,A,B) & = \frac{\sum_{a\in A}f(t,a)}{\vert A\vert} - \frac{\sum_{b\in B}f(t,b)}{\vert B\vert}
\end{align} \noindent then, the association difference between \(A\) a
\(B\) is: \begin{align}
s(X,Y,A,B) & = \sum_{x\in X} s(x,A,B) -  \sum_{y\in Y} s(y,A,B)
\end{align} \noindent \(s(X,Y,A,B)\) is the statistic used in the
significance test, and the \(p\)-value obtained by bootstrapping: it is
the frequency of \(s(X_i,Y_i,A,B)>s(X,Y,A,B)\) for all equally sized
partitions \(X_i, Y_i\) of \(X\cup Y\). The effect size is computed by
normalizing the difference in means as follows: \begin{align}
bias(A,B) & = \frac{
\mu(\{s(x,A,B)\}_{x\in X}) -\mu(\{s(y,A,B)\}_{y\in Y}) 
}{
\sigma(\{s(w,A,B)\}_{w\in X\cup Y})
}
\end{align}

Caliskan et al. (2017) show that significant biases---thus measured---
similar to the ones discovered by IAT can be discovered in word
embeddings. Lauscher \& Glavas (2019) extended the methodology to a
multilingual and cross-lingual setting, arguing that using Euclidean
distance instead of similarity does not make much difference, while the
bias effects vary greatly across embedding models (interestingly, with
social media-text trained embeddings being less biased than those based
on Wikipedia).

A similar methodology is employed by Garg, Schiebinger, Jurafsky, \& Zou
(2018), who employ word embeddings trained on corpora from different
decades to study the shifts in various biases. For instance, to compute
the occupational embeddings bias for women the authors first compute the
average vector of vector embeddings of words that represent women
(e.g.~``she,'' ``female''), then calculate the Euclidean distance
between this mean vector and words for occupations. Then they take the
mean of these distances and subtract from it the analogously obtained
mean for the average vector of vector embeddings of words that represent
men. Formally they take the relative norm distance between \(X\) and
\(Y\) to be: \begin{align}
\textsf{relative norm distance} & = \sum_{v_m\in M} \vert \vert v_m - v_X\vert \vert_2 - \vert v_m - v_Y\vert \vert_2
\end{align} \noindent where the norm used is Euclidean, and \(v_X\) and
\(x_Y\) are average vectors for sets \(X\) and \(Y\) respectively.

Manzini, Lim, Tsvetkov, \& Black (2019) modify WEAT to a multi-class
setting, introducing Mean Average Cosine similarity as a measure of bias
(in fact, in the paper they report distances rather than similarities).
Let \(T = \{t_1, \dots, t_k\}\) be a class of protected word embeddings,
and let each \(A_j\in A\) be a set of attributes stereotypically
associated with a protected word). Then: \begin{align}
S(t_i, A_j) & = \frac{1}{\vert A_j\vert}\sum_{a\in A_j}\mathsf{cos}(t,a) \\
MAC(T,A) & = \frac{1}{\vert T \vert \,\vert A\vert}\sum_{t_i \in T }\sum_{A_j \in A} S(t_i,A_j)
\end{align} That is, for each protected word \$T and each attribute
class, they first take the mean for this protected word and all
attributes in a given attribute class, and then take the mean of thus
obtained means for all the protected words. The t-tests they employ are
run on average cosines used to calculate MAC.

\todo{odleglosc nie musi uchwytywac relacji semantycznej zeby oceniac bias}

\todo{zalozenie, bardziej ze jezeli odleglosci przekladaja sie na downstream, to warto patrzec na odleglosci, nawet bez glebszej filozoficznej interpretacji ich}

\hypertarget{some-methodological-problems}{%
\section{Some methodological
problems}\label{some-methodological-problems}}

Such approaches, however, are not unproblematic. One may consider the
selection of attributes for the bias research. For one thing, the word
lists are often fairly small, and the papers do not discuss the impact
of word list sizes on the uncertainty involved. We will argue that the
word list sizes are too small to yield strong conclusions, given the
variances involved.

Crucially, these approaches use means of mean average cosine
similarities to measure similarity between protected word and harmful
stereotypes. If one takes a closer look at the individual values that
are taken for the calculations, it turns out that there are quite a few
outliers and surprisingly dissimilar words. This issue will become
transparent when we inspect the visualizations of individual cosine
distances, following the idea that one of the first step to understand
data is to look at it.

With such a method the uncertainty involved is not really considered
which makes it even more difficult to give reasonable interpretations of
the results. We propose the use of Bayesian method to obtain some
understanding of the influence the uncertainty has on the interpretation
of final results.

We also observed that in the paper there is no word class distinction
and no control group. In the original paper, words from all three
religions were compared against all of the stereotypes, which means that
there was no distinction between cases in which the stereotype is
associated with a given religion, as opposed to the situation in which
it is associated with another one. Not all of the stereotypical words
have to be considered as harmful for all of the religions. One should
investigate the religions separately as some of them may have stronger
harmful associations that others. One should also include control groups
to have a way of comparing the stereotypical results with neutral or
human-like words. Later in the text we will explain in details reasons
for introducing control groups. In our analysis, we distinguish between
stereotypes associated with a given group, stereotypes associated with
different groups, and control groups: neutral words and stereotypes-free
human predicates.

The interpretation of the results is also a challenge. Assuming for a
moment that the value of multi-class cosine distance is correct, one may
question the interpretation. Manzini et al. (2019) summarize the
averages of cosine distance per group (gender, race, religion). For now
let us focus now on analyzing the values relating to religious biases.
Here is the relevant fragment of table:

\begin{longtable}[]{@{}ll@{}}
\toprule
Religion Debiasing & MAC \\
\midrule
\endhead
Biased & 0.859 \\
Hard Debiased & 0.934 \\
Soft Debiased (\(\lambda\) = 0.2) & 0.894 \\
\bottomrule
\end{longtable}

MAC stand for mean average cosine similarity, although in reality the
the table contains mean distances. What may attract attention is the
fact that the value of cosine distance in ``Biased'' category is already
quite high even before debiasing. High cosine distance indicates low
cosine similarity between values. One could think that average cosine
similarity equal to approximately 0.141 is not significant enough to
consider it as bias. However the authors aim to mitigate ``biases'' in
vectors with such great distance to make it even larger.
Methodologically the question is, on what basis is this small similarity
still considered as a proof of the presence of bias, and whether these
small changes are meaningful. This is in general the problem of scale
and the lack of universal intervals. In contrast, statistical intervals
will help us decide whether a given cosine similarity is high enough to
consider the words to be more similar than if we chose them at random.
We will use highest posterior density intervals, in line with Bayesian
methodology.

One may consider the issue of the curse of dimensionality. In our case,
the curse of dimensionality may take place when there is an increase in
the volume of data that results in adding extra dimensions to the
Euclidean space. As the number of features increases, it may be harder
and harder to obtain useful information from the data using the
available algorithms. Using cosine similarity in high dimensions in word
embeddings may be prone to the curse of dimensionality. According to
(\textbf{Venkat2018Curse?}) there are reasons to consider this
phenomenon when searching for word similarities in higher dimensions. An
experiment is conducted that aims at showing how the similarity values
and variation change as the number of dimensions increases. The
hypothesis made in the paper states that two things will happen as the
number of dimensions increase. First, the effort required to measure
cosine similarity will be greater, and two, the similarity between data
will blur out and have less variation. The authors generate random
points with increasing number of dimensions where each dimension of a
data point is given a value between 0 and 1. Then they pick one vector
at random from each dimension class and calculate the cosine similarity
between the chosen vector and the rest of the data. Then they check how
the variation of values changes as the number of dimensions increases.
It seems like the more dimensions there are, the smaller the variance
and therefore it is less obvious how to interpret the resulting cosine
similarities. Maybe the scale should be adjusted to the number of
dimensions and variance so that it still gives us sensible information
about data.

\todo{MAC: avoid looking at only one vector}

\todo{effect sizes}

\hypertarget{library-for-mac}{%
\section{Library for MAC}\label{library-for-mac}}

\hypertarget{references}{%
\section*{References}\label{references}}
\addcontentsline{toc}{section}{References}

\hypertarget{refs}{}
\begin{CSLReferences}{1}{0}
\leavevmode\vadjust pre{\hypertarget{ref-bolukbasi2016man}{}}%
Bolukbasi, T., Chang, K.-W., Zou, J., Saligrama, V., \& Kalai, A.
(2016). \emph{Man is to computer programmer as woman is to homemaker?
Debiasing word embeddings}. Retrieved from
\url{https://arxiv.org/abs/1607.06520}

\leavevmode\vadjust pre{\hypertarget{ref-Caliskan2017semanticsBiases}{}}%
Caliskan, A., Bryson, J. J., \& Narayanan, A. (2017). Semantics derived
automatically from language corpora contain human-like biases.
\emph{Science}, \emph{356}(6334), 183--186.
\url{https://doi.org/10.1126/science.aal4230}

\leavevmode\vadjust pre{\hypertarget{ref-Garg2018years}{}}%
Garg, N., Schiebinger, L., Jurafsky, D., \& Zou, J. (2018). Word
embeddings quantify 100 years of gender and ethnic stereotypes.
\emph{Proceedings of the National Academy of Sciences}, \emph{115}(16),
E3635--E3644. \url{https://doi.org/10.1073/pnas.1720347115}

\leavevmode\vadjust pre{\hypertarget{ref-Gonen2019lipstick}{}}%
Gonen, H., \& Goldberg, Y. (2019). Lipstick on a pig: {D}ebiasing
methods cover up systematic gender biases in word embeddings but do not
remove them. \emph{Proceedings of the 2019 Conference of the North
{A}merican Chapter of the Association for Computational Linguistics:
Human Language Technologies, Volume 1 (Long and Short Papers)},
609--614. Minneapolis, Minnesota: Association for Computational
Linguistics. \url{https://doi.org/10.18653/v1/N19-1061}

\leavevmode\vadjust pre{\hypertarget{ref-Lauscher2019multidimensional}{}}%
Lauscher, A., \& Glavas, G. (2019). Are we consistently biased?
Multidimensional analysis of biases in distributional word vectors.
\emph{CoRR}, \emph{abs/1904.11783}. Retrieved from
\url{http://arxiv.org/abs/1904.11783}

\leavevmode\vadjust pre{\hypertarget{ref-Manzini2019blackToCriminal}{}}%
Manzini, T., Lim, Y. C., Tsvetkov, Y., \& Black, A. W. (2019).
\emph{Black is to criminal as caucasian is to police: Detecting and
removing multiclass bias in word embeddings}. Retrieved from
\url{https://arxiv.org/abs/1904.04047}

\leavevmode\vadjust pre{\hypertarget{ref-Nissim2020fair}{}}%
Nissim, M., Noord, R. van, \& Goot, R. van der. (2020). Fair is better
than sensational: Man is to doctor as woman is to doctor.
\emph{Computational Linguistics}, \emph{46}(2), 487--497.
\url{https://doi.org/10.1162/coli_a_00379}

\leavevmode\vadjust pre{\hypertarget{ref-Nosek2002harvesting}{}}%
Nosek, B. A., Banaji, M. R., \& Greenwald, A. G. (2002). Harvesting
implicit group attitudes and beliefs from a demonstration web site.
\emph{Group Dynamics: Theory, Research, and Practice}, \emph{6}(1),
101--115. \url{https://doi.org/10.1037/1089-2699.6.1.101}

\end{CSLReferences}

\end{document}
